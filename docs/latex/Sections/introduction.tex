\section{Introducción}

Este trabajo consiste en la aplicación de los conceptos adquiridos en la materia (23.15) Realidad Virtual dictada por el ingeniero Marc S. Ressl. El proyecto consiste en una especie de juego de diseño de interiores donde se inserta al usuario a una sala predefinida por el grupo, y se le da la posibilidad de modificarla a su gusto para poder generar diferentes tipos de diseño sobre una misma distribución de componentes básicos en la sala en cuestión. Esto incluye, poder cambiar objetos, materiales, luces, sonidos, ser capaz de moverse en el espacio asignado y dejar que la imaginación del usuario vuele al intentar encajar de la mejor manera posible la gran cantidad de combinaciones disponibles para los distintos tipos de diseños.
Para adentrarnos un poco mas, en las siguientes secciones se mencionarán con un poco mas de detalle que se puede hacer y como se lo implementó en el trabajo.