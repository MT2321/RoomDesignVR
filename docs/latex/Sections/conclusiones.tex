\section{Conclusiones}

Para finalizar este informe, primero que nada se invita al lector a probar este juego, se encuentra disponible en \url{https://github.com/MT2321/RoomDesignVR} y se lo puede compilar directamente desde ahí, y si quiere directamente la \textit{apk} por favor contactese directamente con alguno de los integrantes de este grupo y se la enviaremos personalmente. Como ultima pata del proyecto nos queda mencionar aquello en lo que pensemos que queramos mejorar o expandirnos para futuras versiones del mismo, ellas se encuentran detalladas aquí debajo:

\begin{itemize}
    \item Agregarle modificación de habitaciones: Actualmente nos enfocamos solo en el living y la cocina, pero para que aparente ser un ambiente real, faltaría agregarle el diseño de las habitaciones y baños.
    \item Añadir mas opciones para cada objeto: Incrementar la capacidad de cambio añadiendo mucha mas cantidad de opciones para cada objeto que se quiera cambiar.
    \item Posibilidad de cambiar materiales a objetos intercambiables: Agregarle la capacidad de cambiar el material a un objeto intercambiable, como por ejemplo los sillones individuales.
    \item Agregarle mas objetos con animaciones.\footnote{Si bien se podrían agregar mas objetos con animaciones como ventiladores, nos costó bastante encontrar una razón suficiente para agregar objetos que se muevan en el diseño de una habitación, es por eso que por el momento se decidió no hacerlo.}
    \item Agregar mas muebles y decoraciones.
    \item Agregar mensajes al usuario indicándole información sobre lo que esta haciendo
    \item Agregarle valor a los objetos con el fin de hacerlo mas dinámico y desafiante al usuario realizar un diseño con un cierto presupuesto dado.
\end{itemize}