% !TEX encoding = UTF-8 Unicode
\documentclass[12pt, a4paper]{article}

\usepackage{todonotes}

% Setup of packages used
\usepackage[onehalfspacing]{setspace}		% One half spacing
\newlength{\stockheight}					% To prevent hyperref warning
\setlength{\stockheight}{\paperheight}		% define \stockheigh
\usepackage{hyperref}					% Hyperlinks on pdf (Should be called before Geometry)
\usepackage[a4paper, 					% Page Layout
                     %showframe,				% This shows the frame
                     includehead,
                     footskip=7mm, headsep=6mm, headheight=4.8mm,
                     marginparsep=2mm, marginparwidth=22mm,
                     top=25mm, bottom=25mm, inner=30mm, outer=25mm]{geometry}
%\usepackage{sansmathfonts}				% Sans Serif equations
\usepackage[T1]{fontenc}	% Output font encoding for internationa

\usepackage[utf8]{inputenc}	% Encoding of files: utf8
 	% Sans Serif as default font
\usepackage{titlesec}					% Redefine chapter and chapter* titles
\titleformat{\chapter}[display]{\huge \bfseries}{\vspace{-0.5cm}\hfill \chaptertitlename\ \thechapter}{0pt}{\hfill}[{\titlerule[1.2pt]}]
\titleformat{name=\chapter,numberless}[display]{\huge \bfseries}{\vspace{-0.5cm}}{0pt}{\hfill}[{\titlerule[1.2pt]}]

% This is used to include the page number on footer within the same margins
%\titleformat{\chapter}[display]{\huge \bfseries}{\vspace{-0.5cm}\hfill \chaptertitlename\ \thechapter}{0pt}{\hfill}[{\titlerule[1.2pt] \enlargethispage{-0.75\baselineskip}}]
%\titleformat{name=\chapter,numberless}[display]{\huge \bfseries}{\vspace{-0.5cm}}{0pt}{\hfill}[{\titlerule[1.2pt] \enlargethispage{-0.75\baselineskip}}]


\usepackage{fancyhdr}					% Package to redefine headers
\fancyhf{}								% No header nor footer
\fancyhead[L]{\thepage}				% Left even and right odd Page Number
\pagestyle{fancy}

\fancypagestyle{plain}{					% To change the footer on chapter and chapter*
	\fancyhf{}							% No header nor footer
%	\fancyfoot[C]{\vspace{-11mm}\thepage}	% Footer with number displaced
	\renewcommand{\headrulewidth}{0pt}	% No line on header
	\renewcommand{\footrulewidth}{0pt}		% No line on footer
}

\RequirePackage{caption} 				% Caption customization
\captionsetup{justification=centerlast,font=small,labelfont=sc,margin=1cm}

\hypersetup{
    colorlinks=true,
    linkcolor=black,
    filecolor=magenta,      
    urlcolor=blue,
    citecolor=black,    
}

% Setup the language and its properties (choose only one)
\usepackage[spanish, es-tabla, es-nodecimaldot]{babel}
\addto\captionsspanish{\renewcommand{\contentsname}{Contenido}}
%\usepackage[english]{babel}
%\addto\captionsenglish{\renewcommand{\contentsname}{Contents}}


%\graphicspath{ {figs/} }					% Use this if custom figures folder is needed

\usepackage{amssymb,amsmath}
\usepackage[square, numbers]{natbib}		% Bibliography
\usepackage{tikz}						% Required for title page
\usetikzlibrary{babel}						% Required to make TikZ works with babel
\usepackage[section]{placeins}				% To flush floats before new section
\usepackage{longtable}					% Used by List of Symbols and friends
\usepackage{array}						% Needed by longtable package

\usepackage{caption}
\usepackage{subcaption}
%para el color del texto
\usepackage{color}
\usepackage{xcolor,colortbl}
\definecolor{LightCyan}{rgb}{0.88,1,1}
%para poder poner H en imagenes
\usepackage{float}

\usepackage{gensymb}

\usepackage{graphicx}
\usepackage[utf8]{inputenc}
\usepackage[export]{adjustbox}
\usepackage{amsmath}

%para poder importar excel
\usepackage{csvsimple}
\usepackage{booktabs}
\usepackage{multirow}
\usepackage{pdfpages}
\usepackage{longtable} 
\usepackage{blindtext}
\usepackage{wrapfig,lipsum,booktabs}


%%Para poder incluir codigo en el informe
\usepackage{listings}
\usepackage{xcolor}

\definecolor{codegreen}{rgb}{0,0.6,0}
\definecolor{codegray}{rgb}{0.5,0.5,0.5}
\definecolor{codepurple}{rgb}{0.58,0,0.82}
\definecolor{backcolour}{rgb}{0.95,0.95,0.92}

\lstdefinestyle{mystyle}{
    backgroundcolor=\color{backcolour},   
    commentstyle=\color{codegreen},
    keywordstyle=\color{magenta},
    numberstyle=\tiny\color{codegray},
    stringstyle=\color{codepurple},
    basicstyle=\ttfamily\footnotesize,
    breakatwhitespace=false,         
    breaklines=true,                 
    captionpos=b,                    
    keepspaces=true,                 
    numbers=left,                    
    numbersep=5pt,                  
    showspaces=false,                
    showstringspaces=false,
    showtabs=false,                  
    tabsize=2
}

\lstset{style=mystyle}


%Table-related commands
\usepackage{array}
\usepackage{adjustbox}
\usepackage[table]{xcolor}


\setlength\parindent{0pt}
% Macros provided
\def\fecha{\ifcase\month\or
  Enero\or Febrero\or Marzo\or Abril\or Mayo\or Junio\or
  Julio\or Agosto\or Septiembre\or Octubre\or Noviembre\or Diciembre\fi \space\number\year
}

%% Para incluir referencias
\usepackage{biblatex}
\addbibresource{biblio.bib}

%%Para que las ecuaciones queden numeradas por sección
\numberwithin{equation}{section}

%% Para que las figuras queden numeradas por sección
\usepackage{chngcntr}
\counterwithin{figure}{section}

\usepackage{caption}
\usepackage{subcaption}

%% Para dibujar circuitos en latex junto con tikz
\usepackage{verbatim}
\usepackage{textcomp}
\usetikzlibrary{decorations.pathmorphing,shapes,arrows,matrix,calc}